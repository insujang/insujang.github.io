%-------------------------
% Resume in Latex
% Author : Sourabh Bajaj
% License : MIT
% Source : https://github.com/sb2nov/resume
%------------------------

\documentclass[letterpaper,oneside,11pt]{article}
\usepackage[left=1in, right=2in, top=1in, bottom=2in, heightrounded]{geometry}

\usepackage{latexsym}
\usepackage[T1]{fontenc}
\usepackage[utf8]{inputenc}
\usepackage{libertine}
\usepackage{titlesec}
\usepackage{marvosym}
\usepackage[usenames,dvipsnames]{color}
\usepackage{verbatim}
\usepackage{enumitem}
\usepackage[hidelinks]{hyperref}
\usepackage{fancyhdr}
\usepackage[english]{babel}
\usepackage{tabularx}
\usepackage{lastpage}
\usepackage{makecell}

% Adjust margins
% \addtolength{\oddsidemargin}{-0.5in}
% \addtolength{\evensidemargin}{-0.5in}
\addtolength{\textwidth}{1in}
% \addtolength{\topmargin}{-.5in}
\addtolength{\textheight}{1.0in}

\pagestyle{fancy}
\fancyhead{} % clear all header and footer fields
\fancyfoot{}
\renewcommand{\headrulewidth}{0pt}
\renewcommand{\footrulewidth}{0pt}
\setlength{\footskip}{24pt}

\fancyfoot[L]{\footnotesize{Insu Jang}}
\fancyfoot[C]{\footnotesize{Page \thepage~of \pageref{LastPage}}}
\fancyfoot[R]{\footnotesize{Last updated: Jul 29, 2023}}


\urlstyle{same}

\raggedbottom
\raggedright
\setlength{\tabcolsep}{0in}

% Sections formatting
\titleformat{\section}{
  \vspace{-4pt}\scshape\raggedright\large
}{}{0em}{}[\color{black}\titlerule \vspace{-5pt}]

%-------------------------
% Custom commands
\newcommand{\resumeItem}[2]{
  \item\small{
    \textbf{#1}{: #2 \vspace{-3pt}}
  }
}
\newcommand{\resumeItemComment}[1]{
  \item\small{
    {#1 \vspace{-2pt}}
  }
}

\newcommand{\resumeSubheading}[4]{
  \vspace{-1pt}\item
    \begin{tabularx}{0.97\textwidth}[t]{X@{\hspace{20pt}}r}
      \textbf{#1} & #2 \\
      \small#3 & \small #4 \\
    \end{tabularx}\vspace{-3pt}
}

\newcommand{\resumeSubItem}[2]{\resumeItem{#1}{#2}\vspace{-5pt}}

\renewcommand{\labelitemii}{$\circ$}
\renewcommand{\cellalign}{l}

\newcommand{\resumeSubHeadingListStart}{\begin{itemize}[leftmargin=*]}
\newcommand{\resumeSubHeadingListEnd}{\end{itemize}}
\newcommand{\resumeItemListStart}{\begin{itemize}}
\newcommand{\resumeItemListEnd}{\end{itemize}\vspace{-5pt}}

%-------------------------------------------
%%%%%%  CV STARTS HERE  %%%%%%%%%%%%%%%%%%%%%%%%%%%%


\begin{document}

%----------HEADING-----------------
\begin{tabular*}{\textwidth}{l@{\extracolsep{\fill}}r}
  \textbf{{\LARGE Insu Jang}} & insujang@umich.edu\\
  4828 BBB, 2260 Hayward Street, Ann Arbor, MI 48109 & \href{https://insujang.github.io}{https://insujang.github.io} \\
\end{tabular*}


\section{Research Interests}
System Architecture, Cloud Computing, Distributed Systems, Heterogeneous Computing, Systems for ML

%-----------EDUCATION-----------------
\section{Education}
  \resumeSubHeadingListStart
  \vspace{-1pt}\item
    \begin{tabularx}{0.97\textwidth}[t]{l@{\extracolsep{\fill}}r}
      \textbf{The University of Michigan} & Ann Arbor, MI, USA \\
      \textit{\small Ph.D. Student in Computer Science and Engineering} & \small Aug 2021 -- Present \\
      \small Advisor: \href{https://www.mosharaf.com}{Dr. Mosharaf Chowdhury}  \\
  \end{tabularx}\vspace{-5pt}
    \vspace{-1pt}\item
      \begin{tabularx}{0.97\textwidth}[t]{l@{\extracolsep{\fill}}r}
        \textbf{Korea Advanced Institute of Science and Technology (KAIST)} & Daejeon, Republic of Korea \\
        \textit{\small Master of Science in Computer Science} & \small Mar 2016 -- Feb 2018 \\
        \small Advisor: \href{http://casys.kaist.ac.kr:8080/~jhuh/}{Dr. Jaehyuk Huh}  \\
        % \multicolumn{2}{X}{\small Thesis: Secure I/O Architecture for Isolated Heterogeneous Computing with Hardware Assisted Trusted Execution Environment} \\
    \end{tabularx}\vspace{-5pt}
    \vspace{-1pt}\item
      \begin{tabularx}{0.97\textwidth}[t]{l@{\extracolsep{\fill}}r}
        \textbf{Sungkyunkwan University (SKKU)} & Seoul, Republic of Korea \\
        \textit{\small Bachelor of Science in Computer Engineering} & \small Mar 2011 -- Feb 2016 \\
    \end{tabularx}\vspace{-5pt}
  \resumeSubHeadingListEnd

%-----------PUBLICATIONS-----------------
\section{Publications}
\begin{enumerate}[leftmargin=*]
  \item \small \textbf{Insu Jang}, Zhenning Yang, Zhen Zhang, Xin Jin, and Mosharaf Chowdhury. \textbf{``Oobleck: Resilient Distributed Training of Large Models Using Pipeline Templates.''} \textit{ACM Symposium on Operating Systems Principles \textbf{(SOSP)}}, October 2023.
  \item \small Jongyul Kim, \textbf{Insu Jang}, Waleed Reda, Jaeseong Im, Marco Canini, Dejan Kostić, Youngjin Kwon, Simon Peter, and Emmett Witchel. \href{https://dl.acm.org/doi/abs/10.1145/3477132.3483565}{\textbf{``LineFS: Efficient SmartNIC Offload of a Distributed File System with Pipeline Parallelism.''}} \textit{ACM Symposium on Operating Systems Principles \textbf{(SOSP)}}, October 2021. \textbf{Best Paper Award.}
  \item \small \textbf{Insu Jang}, Adrian Tang, Taehoon Kim, Simha Sethumadhavan, and Jaehyuk Huh. \href{https://dl.acm.org/doi/abs/10.1145/3297858.3304021}{\textbf{``Heterogeneous Isolated Execution for Commodity GPUs.''}} \textit{International Conference on Architectural Support for Programming Languages and Operating Systems \textbf{(ASPLOS)}}, April 2019.
\end{enumerate}

%-----------RESEARCH EXPERIENCE-----------------
\section{Research Experience}
  \resumeSubHeadingListStart
    \resumeSubheading{Fault Tolerant Distributed Training}{University of Michigan}
    {Studied efficient fault tolerance in large scale distributed training. Implemented Oobleck, a distributed training framework with pre-generated pipeline templates that can recover from failures fast by quickly reinstantiating a pipeline, instead of fully restarting the entire job.
    Oobleck has been accepted to SOSP'23.}{Sep 2021 -- Present}
    
    \resumeSubheading{Offloading Replicated Storage Transactions to RDMA NIC}{KAIST}
    {Reimplemented Hyperloop to use it as a baseline of LineFS, which offloads replicated transaction into Infiniband RDMA adaptors.
    Studied Infiniband RDMA architecture and witnessed the benefits of offloading in reducing host CPU overload. LineFS paper has been published to SOSP'21 and won the best paper award.}{Jan 2020 -- Jul 2020}

    \resumeSubheading{Architectural Support for Trusted Heterogeneous Execution}{KAIST}
    {Designed a HW-SW codesigned architecture for GPU trusted execution environment. To realize it, studied the PCIe interconnect architecture
    and Intel SGX architecture. It focuses on providing protection in the path between the GPU and the CPU to support commodity GPUs for practicality.
    HIX paper has been published to ASPLOS19.}{Mar 2016 -- Feb 2018}

    % \resumeSubheading
    %   {Computer Architecture and Systems Lab, KAIST}{Daejeon, Republic of Korea}
    %   {Research Assistant (Advisor: Dr. Youngjin Kwon)}{Jan 2020 -- Jul 2020}
    %   \resumeItemListStart
    %     \resumeItem{Infiniband RDMA}
    %       {Worked on studying RDMA and reproducing Hyperloop, an Infiniband RDMA framework that optimizes replicated transactions published in SIGCOMM'18.}
    %   \resumeItemListEnd
    % \resumeSubheading
    %   {Computer Architecture and Systems Lab, KAIST}{Daejeon, Republic of Korea}
    %   {Graduate Research Assistant (Advisor: Dr. Jaehyuk Huh)}{Mar 2016 -- Feb 2018}
    %   \resumeItemListStart
    %     \resumeItem{Trusted Heterogeneous Execution Environment}
    %       {Worked on extending the protection of Intel SGX to heterogeneous devices connected to the system via PCIe architecture.}
    %     \resumeItem{Accelerating Machine Learning}
    %       {Worked on studying hardware architecture for machine learning.}
    %   \resumeItemListEnd
    % \resumeSubheading{Neworking Lab, Sungkyunkwan University}{Suwon, Republic of Korea}
    %   {Undergraduate Research Assistant (Advisor: Dr. Hyunseung Choo)}{May 2014 -- Jul 2015}
    %   \resumeItemListStart
    %     \resumeItem{Inaudible Sound Communication System}
    %       {Worked on implementing a data transmission system with inaudible sound over 20kHz frequency using commodity Android smartphones.}
    %   \resumeItemListEnd
    % \resumeSubheading{M2M Lab, Purdue University \footnotesize{(Purdue/NIPA Capstone Program)}}{West Lafayette, IN, USA}
    %   {Undergraduate Research Assistant (Advisor: Dr. Eric T. Matson)}{Jul 2014 -- Aug 2014}
    %   \resumeItemListStart
    %     \resumeItem{Cooprerative Fire Security System using HARMS}
    %       {Worked on implementing Human Agent Robot Machine Sensor (HARMS) message protocol for robot-based firefighting system.}
    %   \resumeItemListEnd
  \resumeSubHeadingListEnd

%-----------PATENTS-----------------
% \section{Patents}
% \begin{enumerate}[leftmargin=*]
%   \item \small \textbf{Insu Jang}, Taehoon Kim, and Jaehyuk Huh. \textbf{``Heterogeneous Isolated Execution for Commodity GPUs.''} \textit{Republic of Korea Patent No. 10-2105760}. Filed June 19, 2018, Issued December 30, 2019.
%   \item \small \textbf{Insu Jang}, Songhee Ryu, Gyuhyeon Jeon, Hyun Lee, Myoungbeom Chung, and Hyunseung Choo. \textbf{``Data Communication Method using Inaudible Frequency Band.''} \textit{Republic of Korea Patent No. 10-1560798}. Filed February 23, 2015, Issued October 16, 2015.
% \end{enumerate}

%-----------WORK EXPERIENCE-----------------
\section{Work Experience}
  \resumeSubHeadingListStart
    \resumeSubheading{Autopilot Software Engineer Intern}{May 2023 -- Aug 2023}{Tesla Inc.}{Palo Alto, CA, USA}

    \resumeSubheading{System Software Engineer}{Feb 2018 -- Jun 2021}
    {TmaxSoft Inc.}{Seongnam, Republic of Korea}

    \resumeSubheading{Research Intern}{Jan 2016 -- Feb 2016}
    {Electronics and Telecommunications Research Institute (ETRI)}{Daejeon, Republic of Korea}

    \resumeSubheading{Research Intern}{Jul 2015 -- Aug 2015}
    {Advanced Institute of Convergence Technology (AICT)}{Suwon, Republic of Korea}

    \resumeSubheading{Student Member}{Jan 2013 -- Apr 2014}
    {Samsung Software Membership \footnotesize{(Student Program of Samsung Electronics)}}{Suwon, Republic of Korea}

    % \resumeSubheading{TmaxSoft Inc.}{Seongnam, Republic of Korea}
    % {}{Feb 2018 -- Jun 2021}
    % % \resumeItemListStart
    % %   \resumeItem{Kubernetes}
    % %     {Worked on implementing a container management system based on Kubernetes.}
    % %   \resumeItem{Linux}
    % %     {Worked on implementing system services using Linux low-level features, e.g. udev, netlink.}
    % %   \resumeItem{Ceph}
    % %     {Worked on analyzing Ceph distributed storage system.}
    % % \resumeItemListEnd

    % % \resumeSubheading{KAIST}{Daejeon, Republic of Korea}{Teaching Assistant (CS230 System Programming)}{Mar 2017 -- Aug 2017}

    % \resumeSubheading{Electronics and Telecommunications Research Institute (ETRI)}{Daejeon, Republic of Korea}
    % {Research Intern}{Jan 2016 -- Feb 2016}
    % % \resumeItemListStart
    % %   \resumeItem{Hypervisor}
    % %     {Worked on studying internal architecture of Xen hypervisor.}
    % %   \resumeItem{Software Development Process}
    % %     {Worked on studying Test Driven Development (TDD).}
    % % \resumeItemListEnd

    % \resumeSubheading{Advanced Institute of Convergence Technology (AICT)}{Suwon, Republic of Korea}
    % {Research Intern}{Jul 2015 -- Aug 2015}
    % % \resumeItemListStart
    % %   \resumeItem{Apache Hadoop and Apache Spark}
    % %     {Worked on studying distributed computing frameworks and internal architecture of Apache Hadoop and Apache Spark.}
    % % \resumeItemListEnd

    % \resumeSubheading{Samsung Software Membership \footnotesize{(Student Program of Samsung Electronics)}}{Suwon, Republic of Korea}
    % {Student Member}{Jan 2013 -- Apr 2014}
    % % \resumeItemListStart
    % %   \resumeItem{Android}
    % %     {Worked on implementing several Android applications.}
    % %   \resumeItem{HTML5 and Javascript}
    % %     {Worked on implementing several HTML5 based web applications.}
    % %   \resumeItemComment{Led several projects as a team leader.}
    % % \resumeItemListEnd
  \resumeSubHeadingListEnd

%-----------AWARDS-----------------
\section{Honors and Awards}
  \resumeSubHeadingListStart
    \resumeSubheading{Best Paper Award}{Oct 2021}
    {\makecell{``LineFS: Efficient SmartNIC Offload of a Distributed File System with Pipeline Parallelism'' \\ The 28th ACM Symposium on Operating Systems Principles (SOSP)}}{}
    \resumeSubheading{Richard H. Orenstein Fellowship in Memory of Murray Orenstein}{Aug 2021}
    {Department of Electrical Engineering and Computer Science, The University of Michigan}{}
    \resumeSubheading{Korea National Scholarship}{Mar 2016}
    {KAIST and Korea Ministry of Science and ICT}{}
    \resumeSubheading{Korea National Scholarship for Science and Engineering}
    {Mar 2014}
    {Korea Student Aid Foundation and Korea Ministry of Education}{}
    \resumeSubheading{2nd Prize, 2015 Convergence App Contest}{Dec 2015}
    {College of Software, Sungkyunkwan University}{}
    \resumeSubheading{Dean's List}{Oct 2014, Apr 2015}
    {Department of Computer Engineering, Sungkyunkwan University}{}
    \resumeSubheading{1st Prize, 2013 Smart TV App and Peripherals Contest}{Nov 2013}
    {Korea Association of Smart Home and Korea Ministry of Trade, Industry and Energy}{}
    \resumeSubheading{1st Prize, 2013 Mobile E-learning App Idea Contest}{Sep 2013}
    {Korea Ministry of Education}{}
  \resumeSubHeadingListEnd

%
%--------PROGRAMMING SKILLS------------
\section{Technical Skills}
 \resumeSubHeadingListStart
  \resumeItem{Languages}{C, C++20, Python, Go, Markdown, \LaTeX}
  \resumeItem{Frameworks}{PyTorch, DeepSpeed, CUDA, Intel SGX, Kubernetes, RDMA, Ceph, Linux, QEMU, KVM}
 \resumeSubHeadingListEnd

\section{References}
Available upon request.
  % \resumeSubHeadingListStart
  %   \resumeItem{Jaehyuk Huh}{Professor, KAIST. jhhuh@kaist.ac.kr}
  %   \resumeItem{Youngjin Kwon}{Assistant Professor, KAIST. yjkwon@kaist.ac.kr}
  %   \resumeItem{Mosharaf Chowdhury}{Morris Wellman Assistant Professor, University of Michigan. mosharaf@umich.edu}
  % \resumeSubHeadingListEnd

\end{document}
