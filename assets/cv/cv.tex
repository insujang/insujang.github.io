%-------------------------
% Resume in Latex
% Author : Sourabh Bajaj
% License : MIT
% Source : https://github.com/sb2nov/resume
%------------------------

\documentclass[letterpaper,oneside,11pt]{article}
\usepackage[left=1in, right=2in, top=1in, bottom=2in, heightrounded]{geometry}

\usepackage{latexsym}
\usepackage[T1]{fontenc}
\usepackage[utf8]{inputenc}
\usepackage{libertine}
\usepackage{titlesec}
\usepackage{marvosym}
\usepackage[usenames,dvipsnames]{color}
\usepackage{verbatim}
\usepackage{enumitem}
\usepackage[hidelinks]{hyperref}
\usepackage{fancyhdr}
\usepackage[english]{babel}
\usepackage{tabularx}
\usepackage{lastpage}
\usepackage{makecell}

% Adjust margins
% \addtolength{\oddsidemargin}{-0.5in}
% \addtolength{\evensidemargin}{-0.5in}
\addtolength{\textwidth}{1in}
% \addtolength{\topmargin}{-.5in}
\addtolength{\textheight}{1.0in}

\pagestyle{fancy}
\fancyhead{} % clear all header and footer fields
\fancyfoot{}
\renewcommand{\headrulewidth}{0pt}
\renewcommand{\footrulewidth}{0pt}
\setlength{\footskip}{24pt}

\fancyfoot[L]{\footnotesize{Insu Jang}}
\fancyfoot[C]{\footnotesize{Page \thepage~of \pageref{LastPage}}}
\fancyfoot[R]{\footnotesize{Last updated: Jan 27, 2026}}


\urlstyle{same}

\raggedbottom
\raggedright
\setlength{\tabcolsep}{0in}

% Sections formatting
\titleformat{\section}{
  \vspace{-4pt}\scshape\raggedright\large
}{}{0em}{}[\color{black}\titlerule \vspace{-5pt}]

%-------------------------
% Custom commands
\newcommand{\resumeItem}[2]{
  \item\small{
    \textbf{#1}{: #2 \vspace{-3pt}}
  }
}
\newcommand{\resumeItemComment}[1]{
  \item\small{
    {#1 \vspace{-2pt}}
  }
}

\newcommand{\resumeSubheading}[4]{
  \vspace{-1pt}\item
    \begin{tabularx}{0.97\textwidth}[t]{X@{\hspace{20pt}}r}
      \textbf{#1} & #2 \\
      \small#3 & \small #4 \\
    \end{tabularx}\vspace{-3pt}
}
\newcommand{\resumeSubheadingFewerItems}[2]{
  \vspace{-1pt}\item
    \begin{tabularx}{0.97\textwidth}[t]{X@{\hspace{20pt}}r}
      \textbf{#1} & #2 \\
    \end{tabularx}\vspace{-3pt}
}

\newcommand{\resumeSubItem}[2]{\resumeItem{#1}{#2}\vspace{-5pt}}
\newcommand{\resumeSubItemWithTime}[2]{\item\small{{#1 \hfill #2}\vspace{-5pt}}}

\renewcommand{\labelitemii}{$\circ$}
\renewcommand{\cellalign}{l}

\newcommand{\resumeSubHeadingListStart}{\begin{itemize}[leftmargin=*]}
\newcommand{\resumeSubHeadingListEnd}{\end{itemize}}
\newcommand{\resumeItemListStart}{\begin{itemize}}
\newcommand{\resumeItemListEnd}{\end{itemize}\vspace{-5pt}}

%-------------------------------------------
%%%%%%  CV STARTS HERE  %%%%%%%%%%%%%%%%%%%%%%%%%%%%


\begin{document}

%----------HEADING-----------------
\begin{tabular*}{\textwidth}{l@{\extracolsep{\fill}}r}
  \textbf{{\LARGE Insu Jang}} & insujang@umich.edu\\
  4828 BBB, 2260 Hayward Street, Ann Arbor, MI 48109 & \href{https://insujang.github.io}{https://insujang.github.io} \\
\end{tabular*}


\section{Research Interests}
Systems for ML, Distributed ML, Large-scale ML Systems, Adaptive Resource Scheduling

%-----------EDUCATION-----------------
\section{Education}
  \resumeSubHeadingListStart
  \vspace{-1pt}\item
    \begin{tabularx}{0.97\textwidth}[t]{l@{\extracolsep{\fill}}r}
      \href{https://umich.edu}{\textbf{The University of Michigan}} & \small Aug 2021 -- May 2026 (Expected) \\ 
      \textit{\small Ph.D. Candidate in Computer Science and Engineering} & \small Ann Arbor, MI, USA \\
      \small Advisor: \href{https://www.mosharaf.com}{Prof. Mosharaf Chowdhury}  \\
  \end{tabularx}\vspace{-5pt}
    \vspace{-1pt}\item
      \begin{tabularx}{0.97\textwidth}[t]{l@{\extracolsep{\fill}}r}
        \href{https://www.kaist.ac.kr/en/}{\textbf{Korea Advanced Institute of Science and Technology (KAIST)}} & \small Mar 2016 -- Feb 2018 \\
        \textit{\small M.Sc. in Computer Science} & \small Daejeon, Republic of Korea \\
        \small Advisor: \href{https://jaehyuk-huh.github.io/}{Prof. Jaehyuk Huh}  \\
        % \multicolumn{2}{X}{\small Thesis: Secure I/O Architecture for Isolated Heterogeneous Computing with Hardware Assisted Trusted Execution Environment} \\
    \end{tabularx}\vspace{-5pt}
    \vspace{-1pt}\item
      \begin{tabularx}{0.97\textwidth}[t]{l@{\extracolsep{\fill}}r}
        \href{https://www.skku.edu/eng/}{\textbf{Sungkyunkwan University (SKKU)}} & \small Mar 2011 -- Feb 2016 \\
        \textit{\small B.Sc. in Computer Engineering} & \small Seoul, Republic of Korea \\
    \end{tabularx}\vspace{-5pt}
  \resumeSubHeadingListEnd

%-----------PUBLICATIONS-----------------
\section{Publications}
\begin{enumerate}[leftmargin=*]
  \item \small \href{https://arxiv.org/abs/2503.11367}{\textbf{Efficient Distributed MLLM Training with Cornstarch}} \\
  \textbf{Insu Jang}, Runyu Lu, Nikhil Bansal, Ang Chen, Mosharaf Chowdhury \\
  arXiv Preprint 2025
  \item \small \href{https://openreview.net/forum?id=ZQiO12xlJq}{\textbf{Mordal: Automated Pretrained Model Selection for Vision Language Models}} \\
  Shiqi He, \textbf{Insu Jang}, Mosharaf Chowdhury \\
  ICLR 2026
  \item \small \href{https://dl.acm.org/doi/abs/10.1145/3694715.3695970}{\textbf{Reducing Energy Bloat in Large Model Training}} \\
  Jae-Won Chung, Yile Gu, \textbf{Insu Jang}, Luoxi Meng, Nikhil Bansal, Mosharaf Chowdhury \\
  ACM SOSP 2024
  \item \small \href{https://dl.acm.org/doi/abs/10.1145/3600006.3613152}{\textbf{Oobleck: Resilient Distributed Training of Large Models Using Pipeline Templates}} \\
  \textbf{Insu Jang}, Zhenning Yang, Zhen Zhang, Xin Jin, Mosharaf Chowdhury \\
  ACM SOSP 2023
  \item \small \href{https://dl.acm.org/doi/abs/10.1145/3477132.3483565}{\textbf{LineFS: Efficient SmartNIC Offload of a Distributed File System with Pipeline Parallelism}} \\
  Jongyul Kim, \textbf{Insu Jang}, Waleed Reda, Jaeseong Im, Marco Canini, Dejan Kostić, Youngjin Kwon, Simon Peter, Emmett Witchel \\
  ACM SOSP 2021 -- \textbf{Best Paper Award!}
  \item \small \href{https://dl.acm.org/doi/abs/10.1145/3297858.3304021}{\textbf{Heterogeneous Isolated Execution for Commodity GPUs}} \\
  \textbf{Insu Jang}, Adrian Tang, Taehoon Kim, Simha Sethumadhavan, Jaehyuk Huh \\
  ACM ASPLOS 2019
\end{enumerate}

%-----------RESEARCH EXPERIENCE-----------------
\section{Research Experience}
  \resumeSubHeadingListStart
    \resumeSubheading{Adaptive Resource Scheduling for Multimodal LLM}{Jan 2024 -- Present}{
      {\textbf{Cornstarch \href{https://github.com/cornstarch-org/Cornstarch}{[source]}}}: A distributed multimodal LLM training framework. 
      It optimizes imbalance across GPUs in pipeline parallelism and context parallelism by exploiting unique characteristics of multimodal LLMs.\newline
      % \textbf{MANDu}: While Cornstarch addresses the imbalance within a batch, multimodal LLM training also introduces cross-batch imbalance due to heterogeneous variability of multimodal inputs.
      % MANDu introduces data-driven dynamic model parallelism to balance the workload across GPUs.
    }{University of Michigan}
    % {\resumeSubHeadingListStart
    % \vspace{-15pt} \item{Cornstarch: efficient resource scheduling for large scale multimodal large language model (MLLM) in training and serving.}\item{Cornserve:}\resumeSubHeadingListEnd}{University of Michigan}

    \resumeSubheading{Fault Tolerant Distributed ML Training}{Sep 2021 -- Oct 2023}
    {\textbf{Oobleck \href{https://github.com/SymbioticLab/Oobleck}{[source]}}: An efficient fault tolerance in large scale distributed training. Oobleck introduces a groundbreaking way of fault tolerance ML; it exploits model states redundancy in data parallelism to recover lost states to avoid restart from the checkpoint, and utilizes every available GPUs by deploying heterogeneous pipeline parallel replicas.}{University of Michigan}
    
    \resumeSubheading{Offloading Operations to RDMA NIC}{Jan 2020 -- Jul 2020}
    {{\textbf{LineFS \href{https://github.com/casys-kaist/LineFS}{[source]}}}: Reimplemented \href{https://dl.acm.org/doi/abs/10.1145/3230543.3230572}{Hyperloop} to use it as a baseline of LineFS, which offloads replicated transaction into Infiniband RDMA adaptors.
    Studied RDMA architecture and witnessed the benefits of offloading in reducing host CPU overload.}{KAIST}

    \resumeSubheading{Architectural Support for Trusted Heterogeneous Execution}{April 2017 -- Oct 2018}
    {\textbf{HIX}: Designed a HW-SW co-design architecture for GPU trusted execution environment. To realize it, studied the PCIe interconnect architecture
    and Intel SGX architecture. It focuses on providing protection in the path between the GPU and the CPU to support commodity GPUs for practicality.}{KAIST}

    % \resumeSubheading
    %   {Computer Architecture and Systems Lab, KAIST}{Daejeon, Republic of Korea}
    %   {Research Assistant (Advisor: Dr. Youngjin Kwon)}{Jan 2020 -- Jul 2020}
    %   \resumeItemListStart
    %     \resumeItem{Infiniband RDMA}
    %       {Worked on studying RDMA and reproducing Hyperloop, an Infiniband RDMA framework that optimizes replicated transactions published in SIGCOMM'18.}
    %   \resumeItemListEnd
    % \resumeSubheading
    %   {Computer Architecture and Systems Lab, KAIST}{Daejeon, Republic of Korea}
    %   {Graduate Research Assistant (Advisor: Dr. Jaehyuk Huh)}{Mar 2016 -- Feb 2018}
    %   \resumeItemListStart
    %     \resumeItem{Trusted Heterogeneous Execution Environment}
    %       {Worked on extending the protection of Intel SGX to heterogeneous devices connected to the system via PCIe architecture.}
    %     \resumeItem{Accelerating Machine Learning}
    %       {Worked on studying hardware architecture for machine learning.}
    %   \resumeItemListEnd
    % \resumeSubheading{Neworking Lab, Sungkyunkwan University}{Suwon, Republic of Korea}
    %   {Undergraduate Research Assistant (Advisor: Dr. Hyunseung Choo)}{May 2014 -- Jul 2015}
    %   \resumeItemListStart
    %     \resumeItem{Inaudible Sound Communication System}
    %       {Worked on implementing a data transmission system with inaudible sound over 20kHz frequency using commodity Android smartphones.}
    %   \resumeItemListEnd
    % \resumeSubheading{M2M Lab, Purdue University \footnotesize{(Purdue/NIPA Capstone Program)}}{West Lafayette, IN, USA}
    %   {Undergraduate Research Assistant (Advisor: Dr. Eric T. Matson)}{Jul 2014 -- Aug 2014}
    %   \resumeItemListStart
    %     \resumeItem{Cooprerative Fire Security System using HARMS}
    %       {Worked on implementing Human Agent Robot Machine Sensor (HARMS) message protocol for robot-based firefighting system.}
    %   \resumeItemListEnd
  \resumeSubHeadingListEnd

%-----------PATENTS-----------------
% \section{Patents}
% \begin{enumerate}[leftmargin=*]
%   \item \small \textbf{Insu Jang}, Taehoon Kim, and Jaehyuk Huh. \textbf{``Heterogeneous Isolated Execution for Commodity GPUs.''} \textit{Republic of Korea Patent No. 10-2105760}. Filed June 19, 2018, Issued December 30, 2019.
%   \item \small \textbf{Insu Jang}, Songhee Ryu, Gyuhyeon Jeon, Hyun Lee, Myoungbeom Chung, and Hyunseung Choo. \textbf{``Data Communication Method using Inaudible Frequency Band.''} \textit{Republic of Korea Patent No. 10-1560798}. Filed February 23, 2015, Issued October 16, 2015.
% \end{enumerate}

%-----------WORK EXPERIENCE-----------------
\section{Work Experience}
  \resumeSubHeadingListStart
  \resumeSubheading{Software Engineering Intern}{May 2025 -- Aug 2025}{ML Networking Team, Google LLC}{Sunnyvale, CA, USA}
  \resumeItemListStart
    \resumeItem{Straggler detection}{Design and implement a framework that systematically analyzes stragglers in distributed ML training.}
    \resumeItem{ML in Kubernetes}{Worked on distributed ML training in Google Kubernetes Engine (GKE).}
  \resumeItemListEnd

    \resumeSubheading{Autopilot Software Engineer Intern}{May 2023 -- Aug 2023}{ML Infrastructure Team, Tesla Inc.}{Palo Alto, CA, USA}
    \resumeItemListStart
      \resumeItem{Straggler detection}{Design core algorithm of detecting stragglers in distributed ML training.}
      \resumeItem{Production deployment}{Implement, deploy, and integrate straggler detection algorithm into the infrastructure. Identified and helped fix several issues.}
    \resumeItemListEnd

    \resumeSubheading{System Software Engineer \footnotesize{-- Fulfillment of Military Obligations}}{Feb 2018 -- Jun 2021}
    {System Kernel Team, TmaxSoft Inc.}{Seongnam, Republic of Korea}
    \resumeItemListStart
      \resumeItem{Network subsystem}{Worked on implementing a network subsystem for TmaxOS.}
      \resumeItem{Virtualization}{Worked on researching virtualization technologies to improve I/O performance.}
      \resumeItem{Ceph \& Kubernetes analysis}{Worked on analyzing Ceph distributed storage system to improve cloud storage performance.}
    \resumeItemListEnd

    % \resumeSubheading{TmaxSoft Inc.}{Seongnam, Republic of Korea}
    % {}{Feb 2018 -- Jun 2021}
    % % \resumeItemListStart
    % %   \resumeItem{Kubernetes}
    % %     {Worked on implementing a container management system based on Kubernetes.}
    % %   \resumeItem{Linux}
    % %     {Worked on implementing system services using Linux low-level features, e.g. udev, netlink.}
    % %   \resumeItem{Ceph}
    % %     {Worked on analyzing Ceph distributed storage system.}
    % % \resumeItemListEnd

    % % \resumeSubheading{KAIST}{Daejeon, Republic of Korea}{Teaching Assistant (CS230 System Programming)}{Mar 2017 -- Aug 2017}

    % \resumeSubheading{Electronics and Telecommunications Research Institute (ETRI)}{Daejeon, Republic of Korea}
    % {Research Intern}{Jan 2016 -- Feb 2016}
    % % \resumeItemListStart
    % %   \resumeItem{Hypervisor}
    % %     {Worked on studying internal architecture of Xen hypervisor.}
    % %   \resumeItem{Software Development Process}
    % %     {Worked on studying Test Driven Development (TDD).}
    % % \resumeItemListEnd

    % \resumeSubheading{Advanced Institute of Convergence Technology (AICT)}{Suwon, Republic of Korea}
    % {Research Intern}{Jul 2015 -- Aug 2015}
    % % \resumeItemListStart
    % %   \resumeItem{Apache Hadoop and Apache Spark}
    % %     {Worked on studying distributed computing frameworks and internal architecture of Apache Hadoop and Apache Spark.}
    % % \resumeItemListEnd

    % \resumeSubheading{Samsung Software Membership \footnotesize{(Student Program of Samsung Electronics)}}{Suwon, Republic of Korea}
    % {Student Member}{Jan 2013 -- Apr 2014}
    % % \resumeItemListStart
    % %   \resumeItem{Android}
    % %     {Worked on implementing several Android applications.}
    % %   \resumeItem{HTML5 and Javascript}
    % %     {Worked on implementing several HTML5 based web applications.}
    % %   \resumeItemComment{Led several projects as a team leader.}
    % % \resumeItemListEnd
  \resumeSubHeadingListEnd

\section{Teaching}
  \resumeSubHeadingListStart
    \resumeSubItemWithTime{TA -- CSE585 Advanced Scalable Systems for Generative AI, The University of Michigan}{Fall 2024}
    \resumeSubItemWithTime{TA -- CS230 System Programming, KAIST}{Spring 2017}
  \resumeSubHeadingListEnd

% \section{Mentoring}
%   \resumeSubHeadingListStart
%   \resumeSubItem{Runyu Lu}{PhD Student @ UM CSE}
%   \resumeSubItem{Kevin Xue}{PhD Student @ UM CSE}
%   \resumeSubItem{Minkyoung Cho}{PhD Student @ UM CSE}
%   \resumeSubItem{Vatsal Joshi}{MS Student @ UM CSE $\rightarrow$ Meta}
%   \resumeSubItem{Luke Zhu}{MS Student @ UM CSE $\rightarrow$ Tesla}
%   \resumeSubHeadingListEnd


%-----------AWARDS-----------------
\section{Honors and Awards}
  \resumeSubHeadingListStart
    \resumeSubheading{Best Paper Award}{Oct 2021}
    {``LineFS: Efficient SmartNIC Offload of a Distributed File System with Pipeline Parallelism''}{}
    \resumeSubheading{Richard H. Orenstein Fellowship in Memory of Murray Orenstein}{Aug 2021}
    {Department of Electrical Engineering and Computer Science, The University of Michigan}{}
    \resumeSubheading{Korea National Scholarship}{Mar 2016}
    {KAIST and Korea Ministry of Science and ICT}{}
    \resumeSubheading{Korea National Scholarship for Science and Engineering}
    {Mar 2014}
    {Korea Student Aid Foundation and Korea Ministry of Education}{}
    % \resumeSubheading{2nd Prize, 2015 Convergence App Contest}{Dec 2015}
    % {College of Software, Sungkyunkwan University}{}
    \resumeSubheading{Dean's List}{Oct 2014, Apr 2015}
    {Department of Computer Engineering, Sungkyunkwan University}{}
    % \resumeSubheading{1st Prize, 2013 Smart TV App and Peripherals Contest}{Nov 2013}
    % {Korea Association of Smart Home and Korea Ministry of Trade, Industry and Energy}{}
    % \resumeSubheading{1st Prize, 2013 Mobile E-learning App Idea Contest}{Sep 2013}
    % {Korea Ministry of Education}{}
  \resumeSubHeadingListEnd

%
%--------PROGRAMMING SKILLS------------
\section{Technical Skills}
 \resumeSubHeadingListStart
  \resumeItem{Languages}{Python, C++, Rust, Triton, English (fluent), Korean (native)}
  \resumeItem{Tools and Frameworks}{PyTorch, Cornstarch, Megatron-LM, DeepSpeed, RDMA, Kubernetes}
 \resumeSubHeadingListEnd

% \section{References}
% Available upon request.
  % \resumeSubHeadingListStart
  %   \resumeItem{Jaehyuk Huh}{Professor, KAIST. jhhuh@kaist.ac.kr}
  %   \resumeItem{Youngjin Kwon}{Assistant Professor, KAIST. yjkwon@kaist.ac.kr}
  %   \resumeItem{Mosharaf Chowdhury}{Morris Wellman Assistant Professor, University of Michigan. mosharaf@umich.edu}
  % \resumeSubHeadingListEnd

\end{document}
